\documentclass[a4paper,12pt]{report}

%%% Работа с русским языком
\usepackage{cmap}					% поиск в PDF
\usepackage{mathtext} 				% русские буквы в фомулах
\usepackage[T2A]{fontenc}			% кодировка
\usepackage[utf8]{inputenc}			% кодировка исходного текста
\usepackage[english,russian]{babel}	% локализация и переносы

%%% Дополнительная работа с математикой
\usepackage{amsfonts,amssymb,amsthm,mathtools} % AMS
\usepackage{amsmath}
\usepackage{icomma} % "Умная" запятая: $0,2$ --- число, $0, 2$ --- перечисление
\usepackage{ dsfont }
%% Номера формул
%\mathtoolsset{showonlyrefs=true} % Показывать номера только у тех формул, на которые есть \eqref{} в тексте.

%% Шрифты
\usepackage{euscript}	 % Шрифт Евклид
\usepackage{mathrsfs} % Красивый матшрифт

%% Свои команды
\DeclareMathOperator{\sgn}{\mathop{sgn}}

%% Перенос знаков в формулах (по Львовскому)
\newcommand*{\hm}[1]{#1\nobreak\discretionary{}
{\hbox{$\mathsurround=0pt #1$}}{}}

%%% Работа с картинками
\usepackage{graphicx}  % Для вставки рисунков
\graphicspath{{images/}{images2/}}  % папки с картинками
\setlength\fboxsep{3pt} % Отступ рамки \fbox{} от рисунка
\setlength\fboxrule{1pt} % Толщина линий рамки \fbox{}
\usepackage{wrapfig} % Обтекание рисунков и таблиц текстом

%%% Работа с таблицами
\usepackage{array,tabularx,tabulary,booktabs} % Дополнительная работа с таблицами
\usepackage{longtable}  % Длинные таблицы
\usepackage{multirow} % Слияние строк в таблице

%%% Заголовок
\author{Похачевский Даниил Андреевич}
\title{2.1 Рисунки и таблицы}
\date{\today}
\title{Линейная алгебра. Второе задание}

\renewcommand*\thesubsection{\arabic{subsection}.}
\renewcommand*\thesection{\Roman{section}.}
 \begin{document}                    % Конец преамбулы, начало текста.
	
%	\maketitle % вывести заголовок, автора
%	\thispagestyle{empty} % не нумеровать первую стран
%	\begin{abstract} % начало аннотации
%		Это наглядное пособие ...
%	\end{abstract} % конец аннотации
% Титульный лист (ГОСТ Р 7.0.11-2001, 5.1)
\thispagestyle{empty}%
\begin{center}%
	\MakeUppercase{министерство образования и науки российской федерации московский физико-технический институт (государственный университет)}
\end{center}%
%
\vspace{0pt plus4fill} %число перед fill = кратность относительно некоторого расстояния fill, кусками которого заполнены пустые места
\begin{flushright}%
	На правах рукописи
\end{flushright}%
%
\vspace{0pt plus6fill} %число перед fill = кратность относительно некоторого расстояния fill, кусками которого заполнены пустые места
\begin{center}%
	{\large Похачевский Даниил Андреевич}
\end{center}%
%
\vspace{0pt plus1fill} %число перед fill = кратность относительно некоторого расстояния fill, кусками которого заполнены пустые места
\begin{center}%
	\textbf {\large ВТОРОЕ ЗАДАНИЕ ПО ЛИНЕЙНОЙ АЛГЕБРЕ \\(2 СЕМЕСТР)}
	
	\vspace{0pt plus2fill} %число перед fill = кратность относительно некоторого расстояния fill, кусками которого заполнены пустые места
	{%\small
		Специальность 10.05.01~---
		
		<<Компьютерная безопасность [б]>>
	}
	
	\vspace{0pt plus2fill} %число перед fill = кратность относительно некоторого расстояния fill, кусками которого заполнены пустые места
	Собственные решения задач из задания
	
\end{center}%
%
\vspace{0pt plus4fill} %число перед fill = кратность относительно некоторого расстояния fill, кусками которого заполнены пустые места

%
\vspace{0pt plus4fill} %число перед fill = кратность относительно некоторого расстояния fill, кусками которого заполнены пустые места
\begin{center}%
	{Иваново~--- 2016}
\end{center}%
	\newpage
	
	\tableofcontents % сгенерировать оглавление
	\thispagestyle{empty}
	\newpage
	\begin{abstract} % начало аннотации
		Это моё второе задание по линейной алгебре. Летом, когда делать было совсем нечего - решил его затехать. Все равно (рано или поздно... лучше раньше)  пришлось бы нарабатывать опыт в \LaTeX. Не знаю, хорошо ли получилось - первокурсники, думаю, оценят. Если кому-нибудь будет интересно, сколько это заняло времени, или кто-то захочет исправить опечатки - пишите письма (мелким почерком) на daniek9898@gmail.com. Желаю успехов в изучении линейной алгебры!
	\end{abstract} % конец аннотации
	\newpage
	\section{Структура линейного преобразования}
	\subsection{Собственные векторы, собственные значения, диагонализируемость}
$$24.13$$
	$det(A - \lambda E)=0	\Longleftrightarrow (\lambda_1 - \lambda)(\lambda_2 - \lambda) \dots (\lambda_{2k+1} - \lambda) = 0$ - 
	\\ многочлен нечетной степени с действительными коэффициентами $\Rightarrow$
	\\$\exists n = \overline{1,{2k+1}} : \lambda_n \in \mathds{R} \Rightarrow \exists$ 
	хотя бы один собственный вектор

$$24.14(1,2)$$
$\chi_\varphi(\lambda)$ - не зависит от выбора базиса.
\\$\vartriangleright$
\\$\varphi \underset{e, e}\longmapsto A$ \qquad $|A - \lambda E| =^{?} |A^{'} - \lambda E|$
\\$ \varphi \underset{e^{'}, e^{'}}\longmapsto A^{'}$  \qquad $A^{'} = S^{-1} A S$
\\ $\underline{|A^{'} - \lambda E|} = |S^{-1} A S - \lambda E| = |S^{-1} A S - S^{-1} (\lambda E) S| = |S^{-1} (A - \lambda E) S| = 
|S^{-1}|\cdot |S| |A - \lambda E| = \underline{|A - \lambda E|} $
\\ $|A - \lambda E| = (-1)^{n} \lambda ^{n} + (-1)^{n-1}\textit{tr}(A) \cdot \lambda^{n-1} + \dots + det(A) $
\\ по т. Виета: $\left \{ 
\begin{gathered} 
\lambda_1 + \lambda_2 + \dots + \lambda_n = tr(A)\\
\lambda_1 \cdot \lambda_2 \dots \cdot \lambda_n = det(A)
\end{gathered}
\right.
$
\begin{flushright}
$\Box$
\end{flushright}

$$24.18(2)$$
\\$L = L^{'} \oplus L^{''}, \lambda, V_{\lambda} - ?$
\\Доказать, что $\varphi$ имеет базис из собственных векторов.
\\2) $\varphi$ - отражение в подпространство $L^{'}  параллельно \, L^{''}$
\\$\forall a \in L$
\\$\varphi: L \longmapsto L \quad \varphi(a) = a_{1} - a_{2}$, где $a_{1} = \textit{proj}_{L'}(a) \, \| L^{''}$, 
$a_{2} \in L^{''}$
\\Пусть $L^{'} = <e_{1}, \dots, e_{k}>, \, L^{''} = <e_{k+1}, \dots, e_{n}>$
\\$L = L^{'} \oplus L^{''} = <e_{1}, \dots, e_{n}>$
\\$\varphi(e_{1}) = e_{1}; \quad \varphi(e_{k+1}) = -e_{k+1};
\\\vdots \qquad \qquad \qquad  \vdots
\\\varphi(e_{k}) = e_{k};	\quad \varphi(e_{n}) = -e_{n}. $
\\
$A_{\varphi} = \begin{bmatrix}

1 & 0 & 0 & 0 & 0 & 0 &\cdots & 0\\

0 & 1 & 0 & 0 & 0 & 0 &\cdots & 0\\

\vdots & \vdots & \ddots & \vdots  & \vdots & \vdots & \vdots & \vdots  \\

0 & 0 & 0 & 1 & 0 & 0 &\cdots & 0\\
0 & 0 & 0 & 0 & -1 &0 & \cdots & 0\\
0 & 0 & 0 & 0 & 0 & -1 &\cdots & 0\\

\vdots & \vdots & \vdots & \vdots & \vdots &\vdots & \ddots & \vdots &\\

0 & 0 & 0 & 0 & 0 & 0 &\cdots & -1\\

\end{bmatrix}$
\\
\\ $\Longrightarrow \exists$ базис из собственных векторов: $e = (e_{1}, \dots, e_{n});$
\\$V_{\lambda_{i}} = <e_{i}>$
$$24.20(3)$$
\\$23.9(3) :\lambda, V_{\lambda}, \textit{diag}(A) - ?$
\\$\varphi$ - ортогональное проектирование $V_{3}$ на $L: x+y+z = 0$
\\$\overrightarrow{n} = 
 \begin{pmatrix}
1 \\ 1 \\ 1
\end{pmatrix}$
\\$\varphi(\overrightarrow{x}) = \overrightarrow{x} - pr_{\overrightarrow{n}}{\overrightarrow{x}} = \overrightarrow{x} - \dfrac{(\overrightarrow{x},\overrightarrow{n})}{\overrightarrow{n}^2}\overrightarrow{n}
$
\\Пусть в $V_{3}$ - ОНБ $e=(e_{1}, e_{2}, e_{3}):$
\\
\\$e_{1} =  
\begin{pmatrix}
1 \\ 0 \\ 0
\end{pmatrix}, \, 
e_{2} =  
\begin{pmatrix}
0 \\ 1 \\ 0
\end{pmatrix}, \,
e_{3} =  
\begin{pmatrix}
0 \\ 0 \\ 1
\end{pmatrix}.$
\\Тогда:
\\$\varphi(e_{1}) =  
\begin{pmatrix}
\dfrac{2}{3} \\[0.25cm] -\dfrac{1}{3} \\[0.25cm] -\dfrac{1}{3}
\end{pmatrix}, \, 
\varphi(e_{2}) =
\begin{pmatrix}
-\dfrac{1}{3} \\[0.25cm] \dfrac{2}{3} \\[0.25cm] -\dfrac{1}{3}
\end{pmatrix}, \,
\varphi(e_{3}) =
\begin{pmatrix}
-\dfrac{1}{3} \\[0.25cm] -\dfrac{1}{3} \\[0.25cm] \dfrac{2}{3}
\end{pmatrix}.$
\\
\\
\\$A_{\varphi} = \dfrac{1}{3}$$\begin{pmatrix}
2 & -1 & -1
\\ -1 & 2&-1
\\-1 & -1 & 2
\end{pmatrix}$
\\	$det(A - \lambda E)=0 \Longleftrightarrow -27(1 - \lambda)^{2}\lambda = 0$
\\

$\left[ 
\begin{gathered} 
\lambda_{1} = 1
\\ \lambda_{2} = 0
\end{gathered}
\right.$ \\ \\
\\1)$\lambda_{1} = 1$\\ 
$A - E = -\dfrac{1}{3} \begin{pmatrix}
1 & 1 & 1
\\ 1 & 1&1
\\1 & 1 & 1
\end{pmatrix}$
$\sim \begin{pmatrix} 1 & 1 & 1 |& 0 \end{pmatrix} \, \Rightarrow$
\\$V_{\lambda_{1}} = <\begin{pmatrix}
-1 \\ 1 \\ 0
\end{pmatrix}, \begin{pmatrix}
-1 \\ 0 \\ 1
\end{pmatrix}>$
\\\;2) $\lambda_{2} = 0 \\
A \sim \begin{pmatrix}
	2 & -1 & -1
	\\ -1 & 2&-1
	\\-1 & -1 & 2
\end{pmatrix} \sim 
\begin{pmatrix}
	1 & 1 & -2
	\\ -2 & 1&1
	\\1 & -2 & 1
\end{pmatrix} 
\sim 
\begin{pmatrix}
1 & 1 & -2
\\ -2 & 1&1
\end{pmatrix} 
\sim 
\begin{pmatrix}
1 & 1 & -2
\\ 0 & 3& -3
\end{pmatrix} 
\sim 
\begin{pmatrix}
1 & 1 & -2
\\ 0 & 1&-1
\end{pmatrix} 
\sim 
\begin{pmatrix}
1 & 0 & -1
\\ 0 & 1&-1
\end{pmatrix} 
\Rightarrow \\
V_{0} = < \begin{pmatrix}
1 \\ 1 \\ 1
\end{pmatrix}>$
\\А в базисе $ \left(
\begin{pmatrix}
-1 \\ 1 \\ 0
\end{pmatrix} ,
\begin{pmatrix}
-1 \\ 0 \\ 1
\end{pmatrix} ,
\begin{pmatrix}
1 \\ 1 \\ 1
\end{pmatrix} \right)$
матрица преобразования имеет вид: \textit{diag}$(1, 1, 0)$

$$24.30(22, 29)$$
$\lambda, \, max \textit{dim}(V_{\lambda}); $ если есть базис из с.в., 
то - матрица преобразования и геометрический смысл
\\22) $\begin{pmatrix}
0 & 0 & 0
\\ 0 & -1&1
\\0 & 0 & -1
\end{pmatrix} 
\\ \chi_{\varphi}(\lambda) = 0 \Longleftrightarrow det
\begin{pmatrix}
-\lambda & 0 & 0
\\ 0 & -1-\lambda&1
\\0 & 0 & -1-\lambda
\end{pmatrix} 
= -\lambda(1+\lambda)^{2} = 0$
\\$\left[ 
\begin{gathered} 
\lambda_{1} = 0
\\ \lambda_{2} = -1
\end{gathered}
\right.$
\\
\\\;1) $\lambda_{1} = 0\\
\begin{pmatrix}
0 & 0 & 0
\\ 0 & -1&1
\\0 & 0 & -1
\end{pmatrix} 
\sim
\begin{pmatrix}
 0 & 1&-1
\\0 & 0 & 1
\end{pmatrix} 
\sim
\begin{pmatrix}
 0 & 1&0
\\0 & 0 & 1
\end{pmatrix} \Longrightarrow
\\V_{\lambda_{0}} = <
\begin{pmatrix}
1 \\ 0 \\ 0
\end{pmatrix} > $,
\, \textit{dim}($ V_{\lambda_{0}} ) = 0$\\ 
\\
\;2) $\lambda_{2} = -1\\
\begin{pmatrix}
1 & 0 & 0
\\ 0 & 0&1
\\0 & 0 & 0
\end{pmatrix}
\sim
 \begin{pmatrix}
 1 & 0 & 0
 \\ 0 & 0&1
 \end{pmatrix}
\Longrightarrow \\
V_{\lambda_{2}} = <
\begin{pmatrix}
0 \\ 1 \\ 0
\end{pmatrix} >, \, \textit{dim}(V_{\lambda_{2}}) = 1 $
$\Longrightarrow $ Нет базиса из собственных векторов.
\\29)
$  \begin{pmatrix}
7 & -12 & 6
\\ 10 & -19&10
\\12 & -24 & 13
\end{pmatrix} \\
\\
\textit{det}(A - \lambda E) = 0 \Longleftrightarrow
\textit{det}
\begin{pmatrix}
7 - \lambda & -12 & 6
\\ 10 & -19- \lambda&10
\\12 & -24 & 13- \lambda
\end{pmatrix} =\\
 = (\lambda - 1)(49 - \lambda^{2} - 120 + 72) = 0 $\\
$\left[ 
\begin{gathered} 
\lambda_{1} = 1
\\ \lambda^{2} = 1
\end{gathered}
\right.
\quad
\left[ 
\begin{gathered} 
\lambda_{1} = 1
\\ \lambda_{2} = -1
\end{gathered}
\right.$
\\1) $\lambda_{1} = 1\\
\begin{pmatrix}
6 & -12 & 6
\\ 10 & -20&10
\\12 & -24 & 12
\end{pmatrix} 
\sim
\begin{pmatrix}
1 & -2 & 1
\end{pmatrix} 
\Longrightarrow
V_{\lambda_{1}} = <
\begin{pmatrix}
2 \\ 1 \\ 0
\end{pmatrix}, \begin{pmatrix}
-1 \\ 0 \\ 1
\end{pmatrix}>
\\2) \lambda_{2} = -1\\
\begin{pmatrix}
8 & -12 & 6
\\ 10 & -18&10
\\12 & -24 & 14
\end{pmatrix} 
\sim
\begin{pmatrix}
4 & -6 & 3
\\ 5 & -9&5
\\6 & -12 & 7
\end{pmatrix} 
\sim
\begin{pmatrix}
8 & -12 & 6
\\ 5 & -9&5
\end{pmatrix}
\sim
\begin{pmatrix}
1 & -3 & 2
\\ 0 & 6&-5
\end{pmatrix}
\sim
\begin{pmatrix}
1 & 0 & -\frac{1}{2}
\\[0.25cm] 0 & 1&-\frac{5}{6}
\end{pmatrix} \Longrightarrow \\
V_{\lambda_{2}}= <\begin{pmatrix}
3 \\ 5 \\ 6
\end{pmatrix}>
\\A = \textit{diag} (1, 1, -1)$ в базисе $\left(
\begin{pmatrix}
2 \\ 1 \\ 0
\end{pmatrix} ,
\begin{pmatrix}
-1 \\ 0 \\ 1
\end{pmatrix} ,
\begin{pmatrix}
3 \\ 5 \\ 6
\end{pmatrix} \right)$
\\
$$24.42(1,2)$$
Оператор дифференцирования D. Найти собственные значения.\\
1) $P^{n} = <1, x, \dots, x^{n}>\\
\, P(x) = a_{n}x^{n} + a_{n-1}x^{n-1}+ \dots + a_{1}x + a_{0} \\
P^{'}(x) = na_{n}x^{n-1} + (n-1)a_{n-1}x^{n-2}+ \dots + a_{1} \\ 
A_{\textit{D}} = \begin{pmatrix}
0 & 1 & 0 & \cdots & 0 & 0 \\
0 & 0 & 2 & \cdots & 0 & 0 \\
\vdots & \vdots & \vdots& \ddots & \vdots & \vdots\\
0 & 0  &\cdots & 0 & n-1 & 0\\
0 & 0 & 0  &\cdots & 0 & n\\
0 & 0 & 0  &\cdots & 0& 0\\
\end{pmatrix} 
\\
\textit{det} (A - \lambda E) = 0 \Longleftrightarrow \textit{det}\begin{pmatrix}
-\lambda & 1 & 0 & \cdots & 0 & 0 \\
0 & -\lambda & 2 & \cdots & 0 & 0 \\
\vdots & \vdots & \ddots& \ddots & \vdots & \vdots\\
0 & 0  &\cdots & -\lambda & n-1 & 0\\
0 & 0 & 0 &\cdots & -\lambda & n\\
0 & 0 & 0  &\cdots & 0& -\lambda\\
\end{pmatrix} = 0 \\
\Longleftrightarrow (-1)^{n}\lambda = 0 
\Longleftrightarrow \lambda = 0$  - собственное значение\\
$\begin{pmatrix}
0 & 1 & 0 & \cdots & 0 & 0 \\
0 & 0 & 2 & \cdots & 0 & 0 \\
\vdots & \vdots & \vdots& \ddots & \vdots & \vdots\\
0 & 0  &\cdots & 0 & n-1 & 0\\
0 & 0 & 0  &\cdots & 0 & n\\
0 & 0 & 0  &\cdots & 0& 0\\
\end{pmatrix} 
\sim
\begin{pmatrix}
0 & 1 & 0 & \cdots & 0 & 0 \\
0 & 0 &1 & \cdots & 0 & 0 \\
\vdots & \vdots & \vdots& \ddots & \vdots & \vdots\\
0 & 0  &\cdots & 0 & 1 & 0\\
0 & 0 & 0  &\cdots & 0 & 1\\
0 & 0 & 0  &\cdots & 0& 0\\
\end{pmatrix} \Longrightarrow
\\V_{\lambda} = <
\begin{pmatrix}
1 \\ 0 \\ \vdots \\ 0
\end{pmatrix} 
>$ - собственное подпространство 
$$24.53$$
$\mathds{M}_{n\times n}; \tau: A \longmapsto A^{T}\\
\tau - $ линейное, $\tau^{2} = \iota; \, \lambda, V_{\lambda}	 $ - ?
\\ Доказать, что: $ \mathds{R}_{n\times n} =  V_{\lambda_{1}} \oplus V_{\lambda_{2}} \oplus \dots \oplus V_{\lambda_{m}}.$
$\\ \triangleright$
линейность - очевидна.
по определению: $A^{T} = \lambda A\\ $
1) на главной диангонали A $\exists a_{ii} \neq 0. $
\\тогда рассмотрим $\lambda = 1: A^{T} = A; $
\\тогда возьмем стандартный базис в пространстве симметрических матриц $n \times n$ 
\\2) $\forall i \, a_{ii} = 0;\\ A^{T} = \lambda A\\$
Пусть $\exists i,j \in \mathds{N}: \\
a=a_{ij}\neq 0; \,b=  a_{ji}\neq 0\\$
Тогда:
\[
\left.
\begin{aligned}
a = a^{*} & =\lambda b\\
b = b^{*} & =\lambda a \\
\end{aligned}
\right\}
\Longrightarrow ab=\lambda^{2} ab \Longrightarrow \lambda^{2}=1 \Longrightarrow \left[ 
\begin{gathered} 
\lambda_{1} = 1
\\ \lambda_{2} = -1
\end{gathered}
\right.\\ \]
\\2.1) $\lambda_{1} = 1$ - рассмотрено ранее\\
2.2) $\lambda_{2} = -1 : A = -A^{T} $ - тогда рассмотрим базис в пространстве кососимметрических матриц $n \times n$.
\\Теперь разберемся с прямой суммой.\\
3)  $ A \in \mathds{M}_{n\times n}, \;  B \in \mathds{M}_{n\times n}^{+}, \; C \in \mathds{M}_{n\times n}^{-}$\\
$ A = B + C\\
\left\{ 
\begin{gathered} 
a_{ij} = b_{ij} + c_{ij}
\\a_{ji} = b_{ij} - c_{ij}
\end{gathered}
\right. \Longrightarrow \begin{gathered} 
b_{ij} = \frac{a_{ij} + a_{ji}}{2}
\\c_{ij} = \frac{a_{ij} - a_{ji}}{2}
\end{gathered}\\
B = \dfrac{A + A^{T}}{2}\\
C = \dfrac{A - A^{T}}{2}$\\
\begin{flushright}
	$\Box$
\end{flushright}
\subsection{Инвариантные подпространства}
$$24.69$$
$U_{1}, U_{2}, \dots, U_{n}$ - инвариантные подпространства относительно $\varphi: \\\varphi(U_{i}) \subset U_{i} $
$\\\vartriangleright (U_{1} + U_{2} + \dots + U_{n}) \supset \varphi(U_{1} + U_{2} + \dots + U_{n}) \\
\, x \in U_{1} \quad \varphi(x) \in U_{1} \quad U_{1} = <e_{1},\dots, e_{k}>\\
\, x \in U_{2} \quad \varphi(x) \in U_{2} \quad U_{1} = <e_{k + 1},\dots, e_{m}>\\
\vdots 	\qquad \qquad \vdots \qquad \qquad \quad \vdots\\
\, x \in U_{n} \quad \varphi(x) \in U_{n} \quad U_{n} = <e_{l+1},\dots, e_{n}>\\
x \in (U_{1} + U_{2} + \dots + U_{n}) \Longleftrightarrow x \in <e_{1},\dots, e_{k}, e_{k + 1},\dots, e_{m}, \dots,e_{l+1},\dots, e_{n} >\\
\varphi(x) = \sum\limits_{i = 1}^n\lambda_{i}\varphi(e_{i}) = \sum\limits_{i = 1}^k\lambda_{i}\varphi(e_{i}) + \dots + \sum\limits_{i = l + 1}^n\lambda_{i}\varphi(e_{i})$, \\где каждая из сумм принадлежит соответственному $U_{i}$\\ $\Longrightarrow  \varphi(x) \in (U_{1} + U_{2} + \dots + U_{n}).\\\\$
\begin{flushright}
	$\Box$
\end{flushright}
$$24.70$$
$\varphi \in L(V, V)$
\\Доказать: $\forall U \leqslant V: Im\varphi \subset U; U$ - инвариантно относительно $\varphi\\
\vartriangleright \forall x \in U: x \in V \Longrightarrow \varphi(x) \subset \varphi(V) = Im\varphi \subset U $
\begin{flushright}
$\Box$
\end{flushright}
\section{Билинейные и квадратичные функции}
$$32.2(3)$$
$x_{1}^2 + 4x_{1}x_{2} + 4x_{1}x_{3} + 5x_{2}^2 + 12x_{2}x_{3} + 7x_{3}^2 $
\\$\begin{pmatrix}
1 & 2 & 2
\\ 2 & 5&6
\\2 & 6 & 7
\end{pmatrix} $
\\$$32.3(5)$$
$\begin{pmatrix}
1 & 1 & 1 &1
\\ 1 & 1&-1&-1
\\1 & -1 & 1& -1
\\1 & -1 & -1&1
\end{pmatrix} $\\
$x_{1}^2 + x_{2}^2 +x_{3}^2 + x_{4}^2 + 2x_{1}x_{2} +  2x_{1}x_{3} +  2x_{1}x_{4} -  2x_{2}x_{3} -  2x_{2}x_{4} - 2x_{3}x_{4}\\$

$$32.7(6)$$
$5x_{1}^2 + 5x_{2}^2 + 3x_{3}^2 + 2x_{1}x_{2} + 2\sqrt{2}x_{1}x_{3} + 2\sqrt{2}x_{2}x_{3}$\\
\\$
	e_{1}^{'} = e_{1} + e_{2} - 2\sqrt{2}e_{3};
	\\e_{2}^{'} = e_{1} - e_{2};
	\\e_{3}^{'} = \sqrt{2}e_{1} + \sqrt{2}e_{2} + e_{3};
$\\
$e^{'} = eS;$\\
\[S = \begin{pmatrix}
1 & 1 & \sqrt{2}
\\ 1 & -1& \sqrt{2}
\\-2\sqrt{2} & 0 & 1
\end{pmatrix}, 
\, B = \begin{pmatrix}
5 & 1 & \sqrt{2}
\\ 1 & 5& \sqrt{2}
\\\sqrt{2} & \sqrt{2} & 3
\end{pmatrix} 
\]
$B^{'} = S^{T}B\overline{S} = \begin{pmatrix}
1 & 1 & -2\sqrt{2}
\\ 1 & -1& 0
\\\sqrt{2} & \sqrt{2} & 1
\end{pmatrix}
\begin{pmatrix}
5 & 1 & \sqrt{2}
\\ 1 & 5& \sqrt{2}
\\\sqrt{2} & \sqrt{2} & 3
\end{pmatrix} 
\begin{pmatrix}
1 & 1 & \sqrt{2}
\\ 1 & -1& \sqrt{2}
\\-2\sqrt{2} & 0 & 1
\end{pmatrix} =\\=\begin{pmatrix}
2 & 2 & -4\sqrt{2}
\\ 4 & -4& 0
\\7\sqrt{2} & 7\sqrt{2} & 7
\end{pmatrix} \begin{pmatrix}
1 & 1 & \sqrt{2}
\\ 1 & -1& \sqrt{2}
\\-2\sqrt{2} & 0 & 1
\end{pmatrix} = \begin{pmatrix}
20 & 0 & 0
\\ 0 & 8& 0
\\0 & 0 & 35
\end{pmatrix}
$
$\\\\B^{'} \underset{e^{'}}\longleftrightarrow 20{x_{1}^{'}}^{2} + 8{x_{2}^{'}}^{2} + 35{x_{3}^{'}}^{2}\\
$
$$32.8(3, 8, 13)$$
$
3)\, -x_{1}x_{2} \longleftrightarrow \begin{pmatrix}
0 & -\frac{1}{2} & 0
\\ -\frac{1}{2} & 0& 0
\\0 & 0 & 0
\end{pmatrix}
\sim
\begin{pmatrix}
0 & 1 & 0
\\ 1 & 0& 0
\\0 & 0 & 0
\end{pmatrix}
\sim
\begin{pmatrix}
2 & 1 & 0
\\ 1 & 0& 0
\\0 & 0 & 0
\end{pmatrix}
\sim
\begin{pmatrix}
2 &0 & 0
\\ 0 & -\frac{1}{2}& 0
\\0 & 0 & 0
\end{pmatrix}
\sim \\ \sim
\begin{pmatrix}
1 &0 & 0
\\ 0 & -1& 0
\\0 & 0 & 0
\end{pmatrix} \longleftrightarrow {x_{1}^{'}}^2 - {x_{2}^{'}}^2\\
8)\, x_{1}^2 + 2x_{1}x_{2} + 2x_{1}x_{3} - 3x_{2}^2 - 6x_{2}x_{3} - 4x_{3}^2\\
\longleftrightarrow \begin{pmatrix}
1 & 1 & 1
\\ 1 & -3& -3
\\1 & -3 & -4
\end{pmatrix} \sim
\begin{pmatrix}
1 & 1 & 0
\\ 1 & -3& 0
\\0 & 0 & -1
\end{pmatrix}
\sim
\begin{pmatrix}
\frac{5}{3} & 0 & 0
\\ 0 & -3& 0
\\0 & 0 & -1
\end{pmatrix}
\sim
\begin{pmatrix}
5 & 0 & 0
\\ 0 & -3& 0
\\0 & 0 & -1
\end{pmatrix}
\longleftrightarrow 5{x_{1}^{'}}^2 - 3{x_{2}^{'}}^2 - {x_{3}^{'}}^2 \\
13) \, x_{1}^2 + 2x_{1}x_{2} + 2x_{2}x_{3} + 2x_{2}^2 + 2x_{4}x_{3} + 2x_{3}^2 + 3x_{4}^2\\
(x_{1} + x_{2})^{2} +(x_{2} + x_{3})^2 + (x_{3} + x_{4})^2 + 2x_{4}^2 = {x_{1}^{'}}^2 + {x_{2}^{'}}^2 + {x_{3}^{'}}^2 + {x_{4}^{'}}^2\\
$
$$32.10(2, 6)$$
$2) x_{1}y_{1} - x_{1}y_{2} - x_{2}y_{1} + x_{2}y_{2} \longleftrightarrow x_{1}(y_{1} - y_{2}) - x_{2}(y_{1} - y_{2}) \longleftrightarrow x_{1}^{'}y_{1}^{'}\\
6)x_{1}y_{1} + 2x_{2}y_{2} + 3x_{3}y_{3} + x_{1}y_{2} + x_{2}y_{1}+x_{1}y_{3}+x_{3}y_{1} + 2x_{2}y_{3} + 2x_{3}y_{2} \longleftrightarrow x_{1}^2 + 2x_{1}x_{2} + 2x_{1}x_{3}+ 2x_{2}x_{3} + 2x_{2}^2 + 3x_{3}^2 =
(x_1 + x_2 + x_3)^2 + x_2^2 + 2x_3^2 = \\= {x_{1}^{'}}^2 + {x_{2}^{'}}^2 + {x_{3}^{'}}^2$
$$32.16$$
Доказать:
$sign\Delta_n = (-1)^n \Longleftrightarrow \text{квадратичная форма отрицательно определена.}
\\\rhd\, $k - отрицательно определена $\Longleftrightarrow$ $b = -k $ - положительно определена\\
$ \exists \,   \underset{e \longrightarrow e^{'}} S : \, k \underset{e^{'}}\longleftrightarrow \begin{pmatrix}
   \gamma_1 & 0 		& \cdots		& 0
\\ 0 		& \gamma_2	& 0 			& \vdots
\\ 0 		& 0 	   	& \ddots		& 0
\\ 0 		& 0			& 0 			& \gamma_n
\end{pmatrix}\\
b = -k \underset{e^{'}}\longleftrightarrow \begin{pmatrix}
-\gamma_1 	& 0 		& \cdots		& 0
\\ 0 		& -\gamma_2	& 0 			& \vdots
\\ 0 		& 0 	   	& \ddots		& 0
\\ 0 		& 0			& 0 			& -\gamma_n
\end{pmatrix}$ - положительно определена $\underset{кр. Сильвестра}\Longleftrightarrow \left\{
\begin{gathered} 
|B_1^{'}| > 0\\
\vdots\\
|B_n^{'}| > 0
\end{gathered} 
\right.$ 
\noindent $|B_1^{'}| > 0 \Longleftrightarrow |B_1| > 0 \Longrightarrow -\gamma_1 > 0 \Longrightarrow |B_1| = \gamma_1 < 0\\
\vdots\\
|B_n^{'}| > 0 \Longleftrightarrow |B_n| > 0 \Longrightarrow (-1)^n \gamma_1\cdots\gamma_n > 0 \Longrightarrow\\\Longrightarrow sign|B_n| = sign(\gamma_1 \cdots \gamma_n) = (-1)^n
$\begin{flushright}
$\Box$
\end{flushright}
\section{Евклидовы пространства}
\subsection{Матрица Грама, ортоганальное дополнение, проекция, ортогонализация}
$$25.7$$
$f, g \in \underset{[-1;1]} C\\
(f,g) = \int\limits_{-1}^{1}fgdt $ - доказать, что это скалярное произведение\\
$\rhd (f,g) = (g, f) - очевидно\\
(f_1 + f_2, g) = \int\limits_{-1}^{1}(f_1 + f_2)gdt = \int\limits_{-1}^{1}f_1gdt + \int\limits_{-1}^{1}f_2gdt = (f_1,g) + (f_2, g)\\
(\lambda f,g) = \int\limits_{-1}^{1}\lambda fgdt = \lambda\int\limits_{-1}^{1}fgdt = \lambda(f,g)
$\begin{flushright}
	$\Box$
\end{flushright}
$$25.25(1)$$
$ a = \begin{pmatrix}
1 \\ 1 \\ 1
\end{pmatrix}, b = \begin{pmatrix}
1 \\ 3 \\ 1
\end{pmatrix}; G = \begin{pmatrix}
   1 & -2& 1
\\ -2& 5& -4
\\ 1 & -4 & 6
\end{pmatrix}\\
(a, b) = X^{T}G\overline{Y} = \begin{pmatrix}
1 & 1 & 1
\end{pmatrix}
\begin{pmatrix}
1 & -2& 1
\\ -2& 5& -4
\\ 1 & -4 & 6
\end{pmatrix}
\begin{pmatrix}
1 \\ 3 \\ 1
\end{pmatrix} = \begin{pmatrix}
0 & -1 & 3
\end{pmatrix}
\begin{pmatrix}
1 \\ 3 \\ 1
\end{pmatrix} = 0
$
$$25.23$$
$
e = (e_1 \dots e_n), \forall x \in V: (x, x) = x_1^2 + x_2^2 + ... + x_n^2\\
Доказать: e - ОНБ\\
\rhd \, \bordermatrix{
	& \cr
	&  0  \cr
	& \vdots  \cr
  k \longrightarrow & 1  \cr
	& \vdots \cr
	& 0  \cr} = e_k,\\ (e_k, e_k) = \begin{pmatrix}
0 & \dots 1& \dots & 0
\end{pmatrix} G \begin{pmatrix}
0 \\ \vdots \\ 1 \\\vdots \\ 0
\end{pmatrix} = \sum\limits_{i = 1}^n g_{ik} = 1 - \forall k = \overline{1,n}
\\
\sum\limits_{i = 1}^n (e_i, e_k) = 1 \Longleftrightarrow (e_1 + \dots + e_{k - 1} + e_{k + 1} + \dots + e_n, e_k) + (e_k, e_k) = 1\\
\Longrightarrow (e_1, e_k) + \dots + (e_{k - 1}, e_k) +(e_{k + 1}, e_k) + \dots + (e_n, e_k) = 0  \\\Longrightarrow(e_i, e_k) = \delta_{ik} \, \forall i,k \in \overline{1,n} \\ \Longrightarrow
G = E_n \Longrightarrow e - ОНБ.
$
\begin{flushright}
	$\Box$
\end{flushright}
$$26.13(4)$$
$e - ОНБ\\
L = <\begin{pmatrix}
4 \\ 3 \\ -3 \\2
\end{pmatrix}, 
\begin{pmatrix}
-1 \\ 3 \\ 2 \\ -3
\end{pmatrix},
\begin{pmatrix}
2 \\ 9 \\ 1 \\ -4
\end{pmatrix}>\\
а) матрица \,L^{\perp} ?\\
б) базис \,L^{\perp}?\\
\rhd а) \, Пусть \, x = \begin{pmatrix}
x_1 \\ x_2 \\ x_3 \\ x_4
\end{pmatrix} \in L^{\perp}\\
\left\{
\begin{gathered} 
(x, a_1) = 0\\
(x, a_2) = 0\\
(x, a_3) = 0\\
(x, a_4) = 0\\
\end{gathered} 
\right. \Longleftrightarrow \begin{pmatrix}
   4 & 3 & -3 & 2
\\ -1 & 3& 2 & -3
\\ 2 & 9 & 1 & -4
\end{pmatrix}\\
\quad б) \begin{pmatrix}
4 & 3 & -3 & 2
\\ -1 & 3& 2 & -3
\\ 2 & 9 & 1 & -4
\end{pmatrix} 
\sim \begin{pmatrix}
0 & 15 & 5 & -10
\\ 2 & 9 & 1 & -4
\end{pmatrix} 
\sim \begin{pmatrix}
2 & 9 & 1 & -4\\
0 & 3 & 1 & -2
\end{pmatrix} 
\sim \begin{pmatrix}
2 & 0 & -2 & -2\\
0 & 3 & 1 & -2
\end{pmatrix} 
\sim \begin{pmatrix}
1 & 0 & -1 & -1\\
0 & 1 & \dfrac{1}{3} & -\dfrac{2}{3}
\end{pmatrix}; \\
базис: \, <  \begin{pmatrix}
3 \\ -1 \\ 3 \\ 0
\end{pmatrix} ,
\begin{pmatrix}
 3 \\ 2 \\ 0 \\ 3
 \end{pmatrix} >.
$
$$26.14(4)$$
$ Найти \,L^{\perp}.\\
L = \begin{pmatrix}
1 & -5 & -6 & 11\\
5 & 1 & -4 & 3\\
1 & 8 & 7 & 15
\end{pmatrix} \\
$Вполне очевидно, что строки порождающей системы будут являться координатными стоблцами векторов, образующих $L^{\perp} $. Поэтому: $
\\ L^{\perp} = <\begin{pmatrix}
1 \\ -5 \\ -6 \\ 11
\end{pmatrix} ,
\begin{pmatrix}
5 \\ 1 \\ -4 \\ 3
\end{pmatrix} , 
\begin{pmatrix}
1 \\ 8 \\ 7 \\ 15
\end{pmatrix}>
$\\
Нужно выделить линейно независимую систему векторов из этой линейной оболочки. Это и будет искомый базис.\\
$\begin{pmatrix}
1 & -5 & -6 & 11\\
5 & 1 & -4 & 3\\
1 & 8 & 7 & 15
\end{pmatrix}
\sim
\begin{pmatrix}
0 & -39 & -39 & -72\\
0 & 13 & 13 & 4\\
1 & 8 & 7 & 15
\end{pmatrix}
\sim
\begin{pmatrix}
1 & 8 & 7 & 15\\
0 & 13 & 13 & 4\\
0 & 0 & 0 & 1
\end{pmatrix}
$ \\$\Longrightarrow$ очевидно, что ранг данной матрицы = 3. Значит, исходная система линейно независимая, и ее можно принять за базисную.
$$26.27(3)$$
$3) e - ОНБ, L \leqslant V
\\a_1 =  \begin{pmatrix}
2 \\ 1 \\ 1 \\ 2
\end{pmatrix}, L = <a_1>\\
x = e\xi; \, \xi = \begin{pmatrix}
5 \\ 3 \\ 7 \\ 0
\end{pmatrix}>.
\\Найти: x^{\perp} \in L^{\perp}, x^{\shortparallel} \in L
\\\rhd x = x^{\perp} + x^{\shortparallel} \\
(x, a_1) = (x^{\shortparallel}, a_1) + (x^{\perp}, a_1) = (x^{\shortparallel}, a_1)\\
x^{\shortparallel} \in L \Longrightarrow x^{\shortparallel} =\alpha a_1\\
(x^{\shortparallel}, a_1) = \alpha(a_1, a_1) = (x, a_1) \Longrightarrow \alpha = \dfrac{(x, a_1)}{(a_1, a_1)} = \dfrac{20}{10} = 2\\
x^{\perp} = x - x^{\shortparallel} = \begin{pmatrix}
1 \\ 1 \\ 5 \\ -4
\end{pmatrix}\\
x^{\shortparallel} = \begin{pmatrix}
4\\ 2 \\ 2 \\ 4
\end{pmatrix} \quad\quad\quad\quad\quad\quad\quad\quad\quad\quad\quad\quad\quad\quad\quad\quad\quad\quad\quad\quad\quad\quad\quad\quad\quad\quad\Box$

$$26.28(4)$$
$4) e - ОНБ, L \leqslant V\\
L = \{ \xi: A\xi = 0\}, \, A =  \begin{pmatrix}
2 & -3 & 5 & 1\\
3 & -5 & 6 & 1
\end{pmatrix}\\
x = e\xi; \, \xi = \begin{pmatrix}
7 \\ -5 \\ 9 \\ 4
\end{pmatrix}.
\\Найти: x^{\perp} \in L^{\perp}, x^{\shortparallel} \in L
\\\rhd
$ очевидно, исходя из задания $L,\, L^{\perp} = <\bordermatrix{
& a_1 \cr
&2 \cr 
&-3 \cr 
& 5 \cr 
& 1 \cr
},
\bordermatrix{
	& a_2 \cr
	&3 \cr 
	&-5 \cr 
	& 6 \cr 
	& 1 \cr
}>\\
x = x^{\perp} + x^{\shortparallel}\\
x^{\perp} = \alpha_1 a_1 + \alpha_2 a_2\\
(x^{\perp}, a_i) = \alpha_1 (a_1, a_i) + \alpha_2 (a_2, a_i) = (x, a_i)\\ \Longrightarrow
\left\{
\begin{gathered} 
(x, a_1) = \alpha_1 (a_1, a_1) + \alpha_2 (a_2, a_1)\\
(x, a_2) = \alpha_1 (a_1, a_2) + \alpha_2 (a_2, a_2)\\
\end{gathered} 
\right. \\
\left( \begin{array}{cc|c}
39 & 52 & 78 \\
52 & 71 & 104
\end{array} \right) \sim
\left( \begin{array}{cc|c}
3 & 4 & 6 \\
52 & 71 & 104
\end{array} \right) \sim
\left( \begin{array}{cc|c}
1 & \dfrac{4}{3} & 2 \\[0.25cm]
1 & \dfrac{71}{52} & 2
\end{array} \right) \sim
\left( \begin{array}{cc|c}
1 & \dfrac{4}{3} & 2 \\[0.25cm]
0 & \dfrac{5}{156} & 0
\end{array} \right) \sim
\left( \begin{array}{cc|c}
1 & 0 & 2 \\
0 & 1 & 0
\end{array} \right)
\\\left\{
\begin{gathered} 
\alpha_{1} = 2
\\ \alpha_{2} = 0
\end{gathered}
\right. \\
x^{\perp} = \begin{pmatrix}
4 \\ -6 \\ 10 \\ 2
\end{pmatrix} \in L^{\perp}\\
x^{\shortparallel} = x - x^{\perp} = \begin{pmatrix}
7 \\ -5 \\ 9 \\ 4
\end{pmatrix} - 
\begin{pmatrix}
4 \\ -6 \\ 10 \\ 2
\end{pmatrix} = \begin{pmatrix}
3 \\ 1 \\ -1 \\ 2
\end{pmatrix} \in L^{\shortparallel}
$ \begin{flushright}
	$\Box$
\end{flushright}
$$26.42(4)$$
$
4) \bordermatrix{
	& e_1 \cr
	&2 \cr 
	&1 \cr 
	& 2 \cr 
}, \bordermatrix{
& e_2 \cr
&6 \cr 
&2 \cr 
& 2 \cr 
}, \bordermatrix{
& e_3 \cr
&1 \cr 
&4 \cr 
& -3 \cr 
} \\ \\
h_1 = e_1\\
h_2 = e_2 - \dfrac{(e_2, h_1)}{(h_1, h_1)}h_1 \\
h_3 = e_3 - \dfrac{(e_3, h_2)}{(h_2, h_2)}h_2 - \dfrac{(e_3, h_1)}{(h_1, h_1)}h_1 \\
h_2 = \begin{pmatrix}
6 \\ 2 \\ 2
\end{pmatrix} - 2 \begin{pmatrix}
2 \\ 1 \\ 2
\end{pmatrix} = 
\begin{pmatrix}
2 \\ 0 \\ -2
\end{pmatrix} \\
h_3 = \begin{pmatrix}
1 \\ 4 \\ -3
\end{pmatrix} - \begin{pmatrix}
2 \\ 0 \\ -2
\end{pmatrix} - 0\begin{pmatrix}
2 \\ 1 \\ 2
\end{pmatrix} = \begin{pmatrix}
-1 \\ 4 \\ -1
\end{pmatrix} \\
\text{Ответ: } \begin{pmatrix}
2 \\ 1 \\ 2
\end{pmatrix},\begin{pmatrix}
2 \\ 0 \\ -2
\end{pmatrix},\begin{pmatrix}
-1 \\ 4 \\ -1
\end{pmatrix}.
$
$$26.44(4)$$
$
4) \bordermatrix{
	& e_1 \cr
	&-3 \cr 
	&2 \cr 
	& 1 \cr 
}, \bordermatrix{
& e_2 \cr
&-8 \cr 
&5 \cr 
& 4 \cr 
}, \bordermatrix{
& e_3 \cr
&2 \cr 
&4 \cr 
& 0 \cr 
}; \quad G = A_{207} = \begin{pmatrix}
1 & 2 & 0 \\
2 & 5 & -2 \\
0 & -2 & 5
\end{pmatrix} \\
h_1 = \begin{pmatrix}
-3 \\ 2 \\ 1
\end{pmatrix}\\
(e_2, h_1) = \begin{pmatrix}
-8 & 5 & 4
\end{pmatrix} 
\begin{pmatrix}
1 & 2 & 0 \\
2 & 5 & -2 \\
0 & -2 & 5
\end{pmatrix}
\begin{pmatrix}
-3 \\ 2 \\ 1
\end{pmatrix} = \begin{pmatrix}
2 & 1 & 10
\end{pmatrix}\begin{pmatrix}
-3 \\ 2 \\ 1
\end{pmatrix} = 6\\
(e_3, h_1) = \begin{pmatrix}
2 & 4 & 0
\end{pmatrix} 
\begin{pmatrix}
1 & 2 & 0 \\
2 & 5 & -2 \\
0 & -2 & 5
\end{pmatrix}
\begin{pmatrix}
-3 \\ 2 \\ 1
\end{pmatrix} = \begin{pmatrix}
10 & 24 & -8
\end{pmatrix}\begin{pmatrix}
-3 \\ 2 \\ 1
\end{pmatrix} = 10\\
(h_1, h_1) = \begin{pmatrix}
-3 & 2 & 1
\end{pmatrix} 
\begin{pmatrix}
1 & 2 & 0 \\
2 & 5 & -2 \\
0 & -2 & 5
\end{pmatrix}
\begin{pmatrix}
-3 \\ 2 \\ 1
\end{pmatrix} = \begin{pmatrix}
1 & 2 & 1
\end{pmatrix}\begin{pmatrix}
-3 \\ 2 \\ 1
\end{pmatrix} = 2\\
h_2 = \begin{pmatrix}
-8 \\ 5 \\ 4
\end{pmatrix} - 3\begin{pmatrix}
-3 \\ 2 \\ 1
\end{pmatrix} =\begin{pmatrix}
1 \\ -1 \\ 1
\end{pmatrix}\\
(h_2, h_2) = \begin{pmatrix}
1 & -1 & 1
\end{pmatrix} 
\begin{pmatrix}
1 & 2 & 0 \\
2 & 5 & -2 \\
0 & -2 & 5
\end{pmatrix}
\begin{pmatrix}
1 \\ -1 \\ 1
\end{pmatrix} = \begin{pmatrix}
-1 & -5 & 7
\end{pmatrix}\begin{pmatrix}
1 \\ -1 \\ 1
\end{pmatrix} = 11\\
(e_3, h_2) = \begin{pmatrix}
2 & 4 & 0
\end{pmatrix} 
\begin{pmatrix}
1 & 2 & 0 \\
2 & 5 & -2 \\
0 & -2 & 5
\end{pmatrix}
\begin{pmatrix}
1 \\ -1 \\ 1
\end{pmatrix} = \begin{pmatrix}
10 & 24 & -8
\end{pmatrix}\begin{pmatrix}
1 \\ -1 \\ 1
\end{pmatrix} = -22\\
h_3 = \begin{pmatrix}
2 \\ 4 \\ 0
\end{pmatrix} + 2\begin{pmatrix}
1 \\ -1 \\ 1
\end{pmatrix} - 5\begin{pmatrix}
-3 \\ 2 \\ 1
\end{pmatrix} = \begin{pmatrix}
19 \\ -8 \\ -3
\end{pmatrix}\\
(h_3, h_3) = \begin{pmatrix}
19 & -8 & -3
\end{pmatrix} 
\begin{pmatrix}
1 & 2 & 0 \\
2 & 5 & -2 \\
0 & -2 & 5
\end{pmatrix}
\begin{pmatrix}
19 \\ -8 \\ -3
\end{pmatrix} = \begin{pmatrix}
3 & 4 & 1
\end{pmatrix}\begin{pmatrix}
19 \\ -8 \\ -3
\end{pmatrix} = 22\\
\widetilde{h_1} = \dfrac{1}{|h_1|}h_1 = \dfrac{1}{\sqrt{2}}\begin{pmatrix}
-3 \\ 2 \\ 1
\end{pmatrix}\\
\widetilde{h_2} = \dfrac{1}{|h_2|}h_2 = \dfrac{1}{\sqrt{11}}\begin{pmatrix}
1 \\ -1 \\ 1
\end{pmatrix}\\
\widetilde{h_3} = \dfrac{1}{|h_3|}h_3 = \dfrac{1}{\sqrt{22}}\begin{pmatrix}
19 \\ -8 \\ -3
\end{pmatrix}\\
$
$$\textbf{T.9}$$
$G = \begin{pmatrix}
1 & 2 & 1 \\
2 & 1 & 1 \\
1 & 1 & 2
\end{pmatrix} - \text{может ли быть матрицей Грамма в каком-либо базисe?} \\
det G < 0 \Rightarrow G  \text{ - не положительно определена, значит не может быть матрицей Грамма}
$ 
\newpage
\subsection{Линейные преобразования евклидовых пространств. Самосопряженные и ортогональные преобразования}
$$28.34(1)$$
$
\text{Пусть } L \subset V \text{ - инвариантно относительно } \varphi \\
\text{Доказать: } L^{\perp} \text{ - инвариантно относительно } \varphi^*\\
\\\rhd
	\varphi^* - \text{сопряженное преобразование для }\varphi \\
	(x, \varphi(y)) = (\varphi^*(x),y)\\
	\forall y \in L: \quad \varphi(y) \subset L \\
	\forall x \in L^{\perp}:\\
	(x, \varphi(y)) = 0 = (\varphi^*(y), y) \Rightarrow \varphi^*(x) \in L^{\perp}	
$ \begin{flushright}
	$\Box$
\end{flushright}
$$\textbf{T.11}$$
$1) A^* = \begin{pmatrix}
1 & 1 \\
-1 & 1 
\end{pmatrix}\\$
Предположим, что данная матрица действительно является матрицей с/c преобразования.
Тогда, по основной теореме о с/с преобразованиях, у него должен существовать ОНБ из собственных векторов.\\
$det \begin{pmatrix}
1 - \lambda & 1 \\
-1 & 1 - \lambda 
\end{pmatrix} = 1 + (1-\lambda)^2 = 0 \text{ - нет корней } \Rightarrow \text{предположение не верно; не может.}\\
2) A = \begin{pmatrix}
2 & 1 \\
-1 & 4 
\end{pmatrix} \\
\chi_{\varphi}(\lambda) = 0\\
(2-\lambda)(4-\lambda) + 1 = 0\\
(\lambda - 3)^2 = 0 \Longleftrightarrow \lambda = 3\\
\text{Найдем собственные вектора:}\\
\lambda = 3:\\
 \begin{pmatrix}
 -1 & 1 \\
 -1 & 1 
 \end{pmatrix} \sim 
  \begin{pmatrix}
  1 & -1
  \end{pmatrix}  \Rightarrow V_{\lambda}=  <\begin{pmatrix}
  1 \\ 1
  \end{pmatrix}> - \text{ собственное подпространство, }\\ dim V_{\lambda} = 1 \\
  \text{Но, алгебраическая кратность корня = 2} \\ \Longrightarrow \nexists \text{ базиса из собственных векторов } \Longrightarrow \text{не может быть матрицей с/с преобразования.}
$
$$29.19(4,8)$$
$\text{Найти матрицу перехода к ОНБ из собственных векторов}
\\4) A_{47} = \begin{pmatrix}
5 & -2 \\
-2 & 8
\end{pmatrix} \\
\chi_{\varphi}(\lambda) = 0 \\
(5 - \lambda)(8 - \lambda) - 4= 0\\
\lambda^2 - 13\lambda + 36 = 0\\
\left[
\begin{gathered}
\lambda_1 = 9\\
\lambda_2 = 4
\end{gathered}
\right. \\
\lambda_1 = 9 : \begin{pmatrix}
-4 & -2 \\
-2 & - 1
\end{pmatrix} \sim
\begin{pmatrix}
1 & \dfrac{1}{2}
\end{pmatrix} \Longrightarrow e_1 = \begin{pmatrix}
1 \\ -2
\end{pmatrix} \\
\lambda_2 = 4 : \begin{pmatrix}
1 & -2 \\
-2 & - 4
\end{pmatrix} \sim
\begin{pmatrix}
1 & -2
\end{pmatrix} \Longrightarrow e_2 = \begin{pmatrix}
2 \\ 1
\end{pmatrix} \\
\text{очевидно, что } e_1 \perp e_2 \\
\text{отнормируем найденный базис: }\\
h_1 = \dfrac{1}{|e_1|}e_1 = \dfrac{1}{\sqrt{1 + 4}}\begin{pmatrix}
1 \\ -2
\end{pmatrix} = \dfrac{1}{\sqrt{5}}\begin{pmatrix}
1 \\ -2
\end{pmatrix}\\
h_2 = \dfrac{1}{|e_2|}e_2 = \dfrac{1}{\sqrt{4 + 1}}\begin{pmatrix}
2 \\ 1
\end{pmatrix} = \dfrac{1}{\sqrt{5}}\begin{pmatrix}
2 \\ 1
\end{pmatrix}\\
S = \dfrac{1}{\sqrt{5}}\begin{pmatrix}
1 & 2 \\
-2 & 1
\end{pmatrix}\\
A^{'} = \begin{pmatrix}
9 & 0 \\
0 & 4
\end{pmatrix} \text{ в базисе }(h_1, h_2)\\
8) A_{280} = \begin{pmatrix}
0 & 2 & 1 \\
2 & 8 & 2\\
1 & 2 & 0
\end{pmatrix} \\
\chi_{\varphi}(\lambda) = 0 \Longleftrightarrow \\
-\lambda(-\lambda(8 - \lambda) - 4) - 2(-2\lambda - 2) + (4 - (8 - \lambda)) = 0\\
-\lambda^3 + 8\lambda^2 + 4\lambda + 4\lambda + 4 +4 -8 + \lambda = 0\\
\lambda^3 - 8\lambda^2 - 9\lambda = 0\\
\lambda(\lambda + 1)(\lambda - 9) = 0\\
\left[
\begin{gathered}
\lambda_1 =  0\\
\lambda_2 = -1\\
\lambda_3 =  9
\end{gathered}
\right. \\
1) \lambda_1 = 0 \\
\begin{pmatrix}
0 & 2 & 1 \\
2 & 8 & 2\\
1 & 2 & 0
\end{pmatrix} \sim
\begin{pmatrix}
1 & 2 & 0 \\
0 & 2 & 1 \\
0 & 4 & 2
\end{pmatrix} \sim
\begin{pmatrix}
1 & 2 & 0 \\
0 & 2 & 1
\end{pmatrix} \sim
\begin{pmatrix}
1 & 0 & -1 \\
0 & 1 & \dfrac{1}{2}
\end{pmatrix} \Longrightarrow e_1 = \begin{pmatrix}
2 \\ -1 \\ 2
\end{pmatrix}\\
2) \lambda_2 = -1 \\
\begin{pmatrix}
1 & 2 & 1 \\
2 & 9 & 2\\
1 & 2 & 1
\end{pmatrix} \sim
\begin{pmatrix}
1 & 2 & 1 \\
0 & 5 & 0 
\end{pmatrix} \sim
\begin{pmatrix}
1 & 0 & 1 \\
0 & 1 & 0
\end{pmatrix} \Longrightarrow e_2 = \begin{pmatrix}
-1 \\ 0 \\ 1
\end{pmatrix}\\ \\
\\ 3) \lambda_3 = 9\\
\begin{pmatrix}
-9 & 2 & 1 \\
2 & -1 & 2\\
1 & 2 & -9
\end{pmatrix} \sim
\begin{pmatrix}
1 & 2 & -9 \\
0 & -5 & 20\\
0 & 20 & -80 
\end{pmatrix} \sim
\begin{pmatrix}
1 & 2 & -9 \\
0 & 1 & -4
\end{pmatrix} \sim 
\begin{pmatrix}
1 & 0 & -1 \\
0 & 1 & -4
\end{pmatrix}
\Longrightarrow e_3 = \begin{pmatrix}
1 \\ 4 \\ 1
\end{pmatrix}\\
\text{У с/с преобразований собственные подпространства взаимно ортоганальны }\\
\text{поэтому остается только отнормировать вектора: } \\
h_1 = \dfrac{1}{|e_1|}e_1 = \dfrac{1}{\sqrt{4 + 1+  4}}\begin{pmatrix}
2 \\ -1 \\ 2
\end{pmatrix} = \dfrac{1}{3}\begin{pmatrix}
2 \\ -1 \\ 2
\end{pmatrix}\\
h_2 = \dfrac{1}{|e_2|}e_2 = \dfrac{1}{\sqrt{1 + 1}}\begin{pmatrix}
-1 \\ 0 \\ 1
\end{pmatrix} = \dfrac{1}{\sqrt{2}}\begin{pmatrix}
-1 \\ 0 \\ 1
\end{pmatrix}\\
h_3 = \dfrac{1}{|e_3|}e_3 = \dfrac{1}{\sqrt{1 + 16 + 1}}\begin{pmatrix}
1 \\ 4 \\ 1
\end{pmatrix} = \dfrac{1}{3\sqrt{2}}\begin{pmatrix}
1 \\ 4 \\ 1
\end{pmatrix}\\
S = \dfrac{1}{3\sqrt{2}}\begin{pmatrix}
2\sqrt{2}  & -3 & 1\\
-1\sqrt{2} & 0  & 4\\
2\sqrt{2}  & 3  & 1
\end{pmatrix}\\
A^{'} = \begin{pmatrix}
0 & 0  & 0\\
0 & -1 & 0 \\
0 & 0  & 9
\end{pmatrix} \text{ в базисе }(h_1, h_2, h_3)\\
$
$$29.47(2,3)$$
$
\varphi \text{ переводит столбцы м. А в столбцы м. B}\\
\varphi - ортоганальное? \\
2)A = A_{44}, B = A_{34}\\
A = \begin{pmatrix}
5  &  -1\\
4  &   1
\end{pmatrix}, \, B = \begin{pmatrix}
2  &   1\\
1  &   1
\end{pmatrix}\\
x \begin{pmatrix}
5 \\ 4
\end{pmatrix} \longrightarrow \varphi(x) \begin{pmatrix}
2 \\ 1
\end{pmatrix}\\
y \begin{pmatrix}
-1 \\ 1
\end{pmatrix} \longrightarrow \varphi(y) \begin{pmatrix}
1 \\ 1
\end{pmatrix} \\
(x, y) = -1; \quad (\varphi(x), \varphi(y)) = 3 \Longrightarrow \varphi - 
\text{ не ортоганальное преобразование}\\
3) A = \begin{pmatrix}
1  &  1 & 3\\
0  &  -1 & -6\\
-1 & 1 & 0
\end{pmatrix}, \, B = \begin{pmatrix}
0  &  1 & 0\\
1  &  -1 & -3\\
-1 & -1 & -6
\end{pmatrix}\\
\left.
\begin{gathered} 
x \begin{pmatrix}
1 \\ 0 \\ -1
\end{pmatrix} \longrightarrow \varphi(x) \begin{pmatrix}
0 \\ 1 \\ -1
\end{pmatrix}\\
y \begin{pmatrix}
1 \\ -1 \\ 1
\end{pmatrix} \longrightarrow \varphi(y) \begin{pmatrix}
1 \\ -1 \\ -1
\end{pmatrix} \\
z \begin{pmatrix}
3 \\ -6 \\ 0
\end{pmatrix} \longrightarrow \varphi(y) \begin{pmatrix}
0 \\ -3 \\ -6
\end{pmatrix} \\
\end{gathered}
\qquad \begin{gathered} 
(x, y) = 0\\
(\varphi(x), \varphi(y)) = 0\\
\\
(y, z) = 9\\
(\varphi(y), \varphi(z)) = 9\\
\\
(x, z) = 3\\
(\varphi(x), \varphi(z)) = 3\\
\end{gathered}
\qquad
\begin{gathered} 
(x, x) = 2\\
(\varphi(x), \varphi(x)) = 2\\
\\
(y, y) = 3\\
(\varphi(y), \varphi(y)) = 3\\
\\
(z, z) = 45\\
(\varphi(z), \varphi(z)) = 45\\
\end{gathered}\\
\right| 
\Longrightarrow\\ \varphi - \text{ортоганальное преобразование}
$
\subsection{Билинейные и квадратичные функции в евклидовых пространствах}
$$32.27(3,19)$$
$3) 7x_1^2 + 4\sqrt{3}x_1x_2 + 3x_2^2 = 7(x_1^2 + 2\dfrac{2\sqrt{3}}{7} x_1x_2 + \dfrac{12}{49}x_2^2) + \dfrac{9}{7}x_2^2 = \\ =7(x_1 + \dfrac{2\sqrt{3}}{7}x_2)^2 + \dfrac{9}{7}x_2^2 = 7{x_1^{'}}^2 + 9{x_2^{'}}^2 \\
A^{'} = \begin{pmatrix}
7 & 0 \\
0 & 9
\end{pmatrix}, S = \begin{pmatrix}
\sqrt{7}			 & 0 \\
\dfrac{2\sqrt{3}}{7} & \dfrac{1}{\sqrt{7}}
\end{pmatrix} \\
19) 2x_1^2 - 4x_1x_2 + 4x_1x_3 + 6x_2^2 - 6x_2x_3 + 2x_2x_4 + 6x_3^2 + 2x_3x_4 + 4x_4^2\\
\text{Будем выполнять элем. преобразования строк-стоблцов} \\
\text{для приведения к диагональному виду}\\
\begin{pmatrix}
2 & -2 & 2  & 0 \\
-2 & 6 & -3 & 1\\
2 & -3 &  6 & 1\\
0 &  1 &  1 & 4 
\end{pmatrix} \sim
\begin{pmatrix}
2 & 0 & 2  & 0 \\
0 & 4 & -1 & 1\\
2 & -1 &  6 & 1\\
0 &  1 &  1 & 4 
\end{pmatrix} \sim
\begin{pmatrix}
2 & 0 & 0  & 0 \\
0 & 4 & -1 & 1\\
0 & -1 & 4 & 1\\
0 &  1 & 1 & 4 
\end{pmatrix} \sim
\begin{pmatrix}
2 & 0 & 0  & 0 \\
0 & 4 & 0 & 1\\
0 & 0 & \dfrac{15}{4} & \dfrac{5}{4}\\[0.25cm]
0 &  1 & \dfrac{5}{4} & 4 
\end{pmatrix} \sim
\begin{pmatrix}
2 & 0 & 0  			  & 0 \\
0 & 4 & 0 			  & 0 \\
0 & 0 & \dfrac{15}{4} & \dfrac{5}{4}\\[0.25cm]
0 & 0 & \dfrac{5}{4}  & \dfrac{15}{4} 
\end{pmatrix} \sim
\begin{pmatrix}
2 & 0 & 0  			  & 0 \\
0 & 4 & 0 			  & 0 \\
0 & 0 & \dfrac{15}{4} & 0 \\
0 & 0 & 0  			  & \dfrac{10}{3} 
\end{pmatrix}\\
$
Для нахождения базиса выполним те же преобразования стобцов над единичной матрицей \\
$S = E S_1 \dots S_k : \begin{pmatrix}
1 & 0 & 0  & 0 \\
0 & 1 & 0 & 0\\
0 & 0 &  1 & 0\\
0 &  0 & 0 & 1 
\end{pmatrix} \sim 
\begin{pmatrix}
1 & 1 & 0  & 0 \\
0 & 1 & 0 & 0\\
0 & 0 &  1 & 0\\
0 &  0 & 0 & 1 
\end{pmatrix} \sim
\begin{pmatrix}
1 & 1 & -1  & 0 \\
0 & 1 & 0 & 0\\
0 & 0 &  1 & 0\\
0 &  0 & 0 & 1 
\end{pmatrix} \sim
\begin{pmatrix}
1 & 1  & -\dfrac{3}{4}   & 0 \\[0.25cm]
0 & 1  &  \dfrac{1}{4}   & 0\\[0.25cm]	
0 & 0  &  1  			 & 0\\
0 & 0  & 0  			 & 1 
\end{pmatrix} \sim
\begin{pmatrix}
1 & 1  & -\dfrac{3}{4}   & - \dfrac{1}{4} \\[0.25cm]
0 & 1  &  \dfrac{1}{4}   &  \dfrac{1}{4}\\[0.25cm]
0 & 0  &  1  			 & 0\\
0 & 0  & 0  			 & 1 
\end{pmatrix} \sim
\begin{pmatrix}
1 & 1  & -\dfrac{3}{4}   & 0 \\[0.25cm]
0 & 1  &  \dfrac{1}{4}   &  \dfrac{1}{6}\\[0.25cm]
0 & 0  &  1  			 & - \dfrac{1}{3}\\[0.25cm]
0 & 0  & 0  			 & 1 
\end{pmatrix} = S
$
$$32.33(4)$$
$
G = A_{207} \\
b : 2x_1^2 + 4x_1x_2 - 2x_1x_3 - x_2^2 + 4x_2x_3 + 2 x_3^2\\
\text{Найти матрицу присоединенного преобразования.} \\
\rhd
A - \text{ матрица присоединенного преобразования к билиннейной функции b} \\
Тогда, \, b(x, y) = (x, A(y)) \Longrightarrow B = GA \Longrightarrow A = G^{-1}B \\
\\B = \begin{pmatrix}
2 & 2 & -1 \\
2 & -1 & 2 \\
-1 & 2 & 2 
\end{pmatrix}, \,G = \begin{pmatrix}
1 & 2 & 0 \\
2 & 2 & -2 \\
0 & -2 & 5 
\end{pmatrix}\\ 
\text{Заметим хитрость: элементарное преобразование строк равносильно }\\
\text{домножению слева на соответствующую элементраную матрицу.}\\ 
\text{Тогда: } (G | B) \underset{элем. преобр.}\rightsquigarrow (E | G^{-1} B) \\
\left( \begin{array}{ccc|ccc}
1 & 2 & 0 & 2 & 2 & -1 \\
2 & 5 & -2 & 2 & -1 & 2 \\
0 & -2 & 5 & -1 & 2 & 2
\end{array} \right) \sim
\left( \begin{array}{ccc|ccc}
1 & 2 & 0 & 2 & 2 & -1 \\
0 & 1 & -2 & -2 & -5 & 4\\
0 & 1 & -\dfrac{5}{2} & \dfrac{1}{2} & -1 & -1 
\end{array} \right) \sim\\ \sim
\left( \begin{array}{ccc|ccc}
1 & 2 & 0 & 2 & 2 & -1 \\
0 & 1 & -2 & -2 & -5 & 4\\
0 & 0 & -\dfrac{1}{2} & \dfrac{5}{2} & 4 & -5 
\end{array} \right) \sim
\left( \begin{array}{ccc|ccc}
1 & 2 & 0 & 2 & 2 & -1 \\
0 & 1 & 0 & -12 & -21 & 24\\
0 & 0 & 1 & -5 & -8 & 10 
\end{array} \right) \sim \\ \sim
\left( \begin{array}{ccc|ccc}
1 & 0 & 0 & 26 & 44 & -49 \\
0 & 1 & 0 & -12 & -21 & 24\\
0 & 0 & 1 & -5 & -8 & 10 
\end{array} \right) \\
Ответ: \, A = \begin{pmatrix}
26 & 44 & -49 \\
-12 & -21 & 24\\
-5 & -8 & 10 
\end{pmatrix}
$ \begin{flushright}
	$\Box$
\end{flushright}
$$32.36(5,8)$$
$5) \; f = x_1^2 - 2x_1x_2 + x_2^2 \\
g = 17 x_1^2 + 8x_1x_2 +x_2^2 \\
f: \begin{pmatrix}
1 & -1 \\
-1 & 1
\end{pmatrix}, g : \begin{pmatrix}
17 & 4 \\
4 & 1
\end{pmatrix}\\
Миноры \; g: 17; 1 \Longrightarrow g - положительно \; определена\\
Примем \; g \; за \; м.Грамма \;- G\\
det(F - \lambda G) = 0 \; - \text{  обобщенное характеристическое уравнение}\\
det \begin{pmatrix}
1 - 17\lambda & -1 - 4\lambda \\
- 1 - 4\lambda & 1 - \lambda
\end{pmatrix} = 1 - 17\lambda - \lambda + 17\lambda^2 - 1 - 16\lambda^2 - 8\lambda = \lambda(\lambda - 26) \\
\left[
\begin{gathered}
\lambda_1 =  0\\
\lambda_2 = 26
\end{gathered}
\right. \\
1)\lambda_1 = 0: \begin{pmatrix}
1 & -1 \\
-1 & 1
\end{pmatrix} \sim
\begin{pmatrix}
1 & -1 
\end{pmatrix} \Longrightarrow a_1 = \begin{pmatrix}
1 \\ 1
\end{pmatrix}\\
2)\lambda_2 = 26: \begin{pmatrix}
1 - 17 \cdot 26 & -1 - 4 \cdot 26 \\
-1 - 4 \cdot 26 & 1 - 26
\end{pmatrix} \sim
\begin{pmatrix}
-105 & -25
\end{pmatrix} \Longrightarrow a_2 = \begin{pmatrix}
-5 \\ 21
\end{pmatrix}\\
(a_1, a_1) = \begin{pmatrix}
1 & 1
\end{pmatrix} \begin{pmatrix}
17 & 4 \\
4 & 1 \\
\end{pmatrix} \begin{pmatrix}
1 \\ 1
\end{pmatrix} = \begin{pmatrix}
21 & 5
\end{pmatrix} \begin {pmatrix}
1 \\ 1
\end{pmatrix} = 26\\
(a_2, a_2) = \begin{pmatrix}
-5 & 21
\end{pmatrix} \begin{pmatrix}
17 & 4 \\
4 & 1 \\
\end{pmatrix} \begin{pmatrix}
-5 \\ 21
\end{pmatrix} = \begin{pmatrix}
-1 & -1
\end{pmatrix} \begin {pmatrix}
-5 \\ 21
\end{pmatrix} = 26\\ 
a_1^{'} = \dfrac{a_1}{|a_1|} = \dfrac{1}{\sqrt{26}}\begin{pmatrix}
1 \\ 1
\end{pmatrix} \\
a_2^{'} = \dfrac{a_2}{|a_2|} = \dfrac{1}{\sqrt{26}}\begin{pmatrix}
-5 \\ 21
\end{pmatrix}\\
\\ \text{В этом базисе:}\\
F^* \longleftrightarrow \begin{pmatrix}
0 & 0 \\
0 & 26
\end{pmatrix} \\
G^* \longleftrightarrow \begin{pmatrix}
1 & 0 \\
0 & 1
\end{pmatrix} \\
S = \dfrac{1}{\sqrt{26}}\begin{pmatrix}
1 & -5 \\
1 & 21
\end{pmatrix} \\
8) \; f = (1 + 2m\sqrt{a^2 + a})x_1^2 + 2\sqrt{a^2 + a}x_1x_2\\
g = (1+ m^2)x_1^2 + 2mx_1x_2 + x_2^2, \; m, a \in \mathbb{R}; \; a^2 + a \geqslant 0 \\
\\f: \begin{pmatrix}
1 + 2m\sqrt{a^2 + a} & \sqrt{a^2 + a} \\
\sqrt{a^2 + a} &  0
\end{pmatrix}, g : \begin{pmatrix}
1 + m^2 & m \\
m & 1
\end{pmatrix}\\
Миноры \; g: 1 + m^2; 1 \Longrightarrow g - положительно \; определена\\
Примем \; g \; за \; м.Грамма \;- G\\\\
det(F - \lambda G) = 0 \; - \text{  обобщенное характеристическое уравнение}\\
det \begin{pmatrix}
1 + 2m\sqrt{a^2+a} -\lambda -\lambda m^2 & \sqrt{a^2 + a} -\lambda m\\
\sqrt{a^2 + a} - \lambda m & - \lambda
\end{pmatrix} =-\lambda - 2m \lambda \sqrt{a^2 + a} + \lambda^2(1+m^2) - (a^2 + a + \lambda^2 m^2 - 2\lambda m \sqrt{a^2 +a }) =  \lambda^2 -\lambda - a^2 - a = \\ = (\lambda - a)(\lambda + a) - (\lambda + a) = (\lambda + a)(\lambda - a - 1)\\
\left[
\begin{gathered}
\lambda_1 =  -a\\
\lambda_2 = a + 1
\end{gathered}
\right. \\ 
1)\lambda_1 = -a: \begin{pmatrix}
1 + 2m\sqrt{a^2+a} +a(1 + m^2) & \sqrt{a^2 + a}  +a m\\
\sqrt{a^2 + a} +a m & a
\end{pmatrix} \sim \begin{pmatrix}
\sqrt{a^2 + a} +a m & a
\end{pmatrix} \Longrightarrow a_1 = \begin{pmatrix}
-a \\ \sqrt{a^2 + a} +a m
\end{pmatrix} \\
2) \lambda_2 = a + 1: \begin{pmatrix}
1 + 2m\sqrt{a^2+a} -(a+1)(1 + m^2) & \sqrt{a^2 + a} - (a+1) m\\
\sqrt{a^2 + a} - (a+1) m & - (a+1)
\end{pmatrix} \sim
\begin{pmatrix}
\sqrt{a^2 + a} - m(a+1) & - (a+1)
\end{pmatrix} \Longrightarrow a_2 = \begin{pmatrix}
a+1 \\ \sqrt{a^2 + a} - m(a+1)
\end{pmatrix} \\
(a_1, a_1) = \begin{pmatrix}
-a & \sqrt{a^2 + a} +a m
\end{pmatrix} \begin{pmatrix}
1 + m^2 & m \\
m & 1
\end{pmatrix} \begin{pmatrix}
-a \\ \sqrt{a^2 + a} +a m
\end{pmatrix} = \\ = \begin{pmatrix}
-a + m\sqrt{a^2 + a} & \sqrt{a^2 + a}
\end{pmatrix} \begin{pmatrix}
-a \\ \sqrt{a^2 + a} +a m
\end{pmatrix} = a(2a + 1)\\
\\(a_2, a_2) = \begin{pmatrix}
a+1 & \sqrt{a^2 + a} - m(a+1)
\end{pmatrix} \begin{pmatrix}
1 + m^2 & m \\
m & 1
\end{pmatrix} \begin{pmatrix}
a+1 \\ \sqrt{a^2 + a} - m(a+1)
\end{pmatrix} = \\ = \begin{pmatrix}
a+1 + m\sqrt{a^2 + a} & \sqrt{a^2 + a}
\end{pmatrix} \begin{pmatrix}
a+1 \\ \sqrt{a^2 + a} - m(a+1)
\end{pmatrix} = 2a^2 + 3a + 1 = (a+1)(2a+1)\\
a_1^{'} = \dfrac{a_1}{|a_1|} = \dfrac{1}{\sqrt{a(2a+1)}}\begin{pmatrix}
-a \\ \sqrt{a^2 + a} +a m
\end{pmatrix} \\
a_2^{'} = \dfrac{a_2}{|a_2|} = \dfrac{1}{\sqrt{(a+1)(2a+1)}}\begin{pmatrix}
a+1 \\ \sqrt{a^2 + a} - m(a+1)
\end{pmatrix} \\
\\ \text{В этом базисе:}\\
F^* \longleftrightarrow \begin{pmatrix}
-a & 0 \\
0 & a+1
\end{pmatrix} \\
G^* \longleftrightarrow \begin{pmatrix}
1 & 0 \\
0 & 1
\end{pmatrix} \\
S = \dfrac{1}{\sqrt{2a+1}}\begin{pmatrix}
-\sqrt{a} & \sqrt{a+1} \\
\sqrt{a+1} + m\sqrt{a} & \sqrt{a} - m\sqrt{a+1}
\end{pmatrix} \\
$\\
\\$$32.39(2)$$
$f = 89x_1^2 - 42x_1x_2 + 5x_2^2 \longleftrightarrow \begin{pmatrix}
89 & -21\\
-21 & 5
\end{pmatrix} \\
g = 41x_1^2 - 18x_1x_2 + 2x_2^2 \longleftrightarrow \begin{pmatrix}
41 & -9\\
-9 & 2
\end{pmatrix} \\
Миноры \; g: 41; 1 \Longrightarrow g - положительно \; определена\\
Примем \; g \; за \; м.Грамма \;- G\\
det(F - \lambda G) = 0 \; - \text{  обобщенное характеристическое уравнение}\\
det \begin{pmatrix}
89 - 41\lambda & -21 + 9\lambda\\
-21 + 9\lambda & 5 - 2\lambda
\end{pmatrix} = 0 \Longleftrightarrow \\
(89-41\lambda)(5 - 2\lambda) - (9\lambda - 21)^2 = 0 \\
\lambda^2 - 5\lambda + 4 = 0 \\
\left[
\begin{gathered}
\lambda_1 =  1\\
\lambda_2 = 4
\end{gathered}
\right. \\ 
F^* \longleftrightarrow \begin{pmatrix}
1 & 0 \\
0 & 4
\end{pmatrix} \\
G^* \longleftrightarrow \begin{pmatrix}
1 & 0 \\
0 & 1
\end{pmatrix} \\
$
\end{document}






























